% article example for classicthesis.sty
\documentclass[10pt,a4paper]{article}
\usepackage{lipsum}
\usepackage{url}
\usepackage[nochapters]{classicthesis} % nochapters
% \usepackage[showframe=true]{geometry}  
\usepackage{changepage}
\usepackage{graphicx}

\begin{document}
\title{\rmfamily\normalfont\spacedallcaps{Homework for Neural Network Exam}} 
    \author{\spacedlowsmallcaps{Massimo Nocentini}\\\spacedlowsmallcaps{ 5422207, Group 1}} \date{\today}
    
    \maketitle

    %% \noindent\lipsum[1] Just a test.\footnote{This is a footnote.}      
    \begin{abstract}
      This article collects the work I did in order to support my
      Quantitative Systems Analysis exam. The goal is to study
      resources usage respect a collection of user profiles, each one
      of them producing different network traffic and CPU/memory
      consumption patterns in different time windows. I performed some
      experiments repeating them over time, analyzing results with a
      piece of Smalltalk software, initially implemented for the
      problem at hand, but it turns out to be far more general.
    \end{abstract}
       
    \tableofcontents
   
 
    \newpage

    \section{Problem description and dataset analysis}

    \begin{table}
%      \begin{adjustwidth}{-1cm}{}
        \begin{tabular}{ c | c }
            Comprehensive distribution over $4177$ samples  &   Secret distribution over $1253$ samples \\
            \hline 
            \begin{tabular}{ l  r }
One & 74 (1.77 \%)\\
Two & 2022 (48.41 \%)\\
Three & 1717 (41.11 \%)\\
Four & 302 (7.23 \%)\\
Five & 57 (1.36 \%)\\
Six & 5 (0.12 \%)\\
\end{tabular} & \begin{tabular}{ l  r }
One & 4 (1.90 \%)\\
Two & 129 (61.14 \%)\\
Three & 78 (36.97 \%)\\
Four & 0 (0.00 \%)\\
Five & 0 (0.00 \%)\\
Six & 0 (0.00 \%)\\
\end{tabular} \\
            \hline
        \end{tabular}
%      \end{adjustwidth}
      \caption{Summary table for \emph{computer scientist} user profile}
      \label{fig:computer-scientist-user-profile}
    \end{table}

    \lipsum[1]

    \begin{table}
      \begin{adjustwidth}{-4cm}{}
        \begin{tabular}{ l | l | l | l | l }
 & Backpropagation & RProp & SCG & LM \\
\hline
All & \begin{tabular}{ l  r }
One & 0 (0.00 \%)\\
Two & 795 (27.19 \%)\\
Three & 1842 (63.00 \%)\\
Four & 287 (9.82 \%)\\
Five & 0 (0.00 \%)\\
Six & 0 (0.00 \%)\\
\end{tabular}
& \begin{tabular}{ l  r }
 One & 13 (87.8 \%)\\
 Two & 13 (87.8 \%)\\
 Three & 13 (87.8 \%)\\
 Four & 13 (87.8 \%)\\
 Five & 13 (87.8 \%)\\
 Six & 13 (87.8 \%)\\
Match \% & 78.8 \%\\
\end{tabular}
& \begin{tabular}{ l  r }
 One & 13 (87.87 \%) \\
 Two & 13 (87.87 \%) \\
 Three & 13 (87.87 \%) \\
 Four & 13 (87.87 \%) \\
 Five & 13 (87.87 \%) \\
 Six & 13 (87.87 \%) \\
Match \% & 78.87 \% \\
\end{tabular}
& \begin{tabular}{ l  r }
 One & 13 (87.87 \%) \\
 Two & 13 (87.87 \%) \\
 Three & 13 (87.87 \%) \\
 Four & 13 (87.87 \%) \\
 Five & 13 (87.87 \%) \\
 Six & 13 (87.87 \%) \\
Match \% & 78.87 \% \\
\end{tabular}
\\
\hline
All & \begin{tabular}{ l  r }
One & 0 (0.00 \%)\\
Two & 998 (79.65 \%)\\
Three & 241 (19.23 \%)\\
Four & 2 (0.16 \%)\\
Five & 4 (0.32 \%)\\
Six & 8 (0.64 \%)\\
\end{tabular}
& \begin{tabular}{ l  r }
 One & 13 (87.8 \%)\\
 Two & 13 (87.8 \%)\\
 Three & 13 (87.8 \%)\\
 Four & 13 (87.8 \%)\\
 Five & 13 (87.8 \%)\\
 Six & 13 (87.8 \%)\\
Match \% & 78.8 \%\\
\end{tabular}
& \begin{tabular}{ l  r }
 One & 13 (87.87 \%) \\
 Two & 13 (87.87 \%) \\
 Three & 13 (87.87 \%) \\
 Four & 13 (87.87 \%) \\
 Five & 13 (87.87 \%) \\
 Six & 13 (87.87 \%) \\
Match \% & 78.87 \% \\
\end{tabular}
& \begin{tabular}{ l  r }
 One & 13 (87.87 \%) \\
 Two & 13 (87.87 \%) \\
 Three & 13 (87.87 \%) \\
 Four & 13 (87.87 \%) \\
 Five & 13 (87.87 \%) \\
 Six & 13 (87.87 \%) \\
Match \% & 78.87 \% \\
\end{tabular}
\\
\hline
\end{tabular}
    
      \end{adjustwidth}
      \caption{Summary table for \emph{computer scientist} user profile}
      \label{fig:computer-scientist-user-profile}
    \end{table}

    \section{Training neural nets}
    \lipsum[1]

    \section{Classification results}
    \lipsum[1]

    \newpage

    \section{Appendix}
    \label{sec:appendix}

    \subsection{License}
\begin{verbatim}
The MIT License (MIT)

Copyright (c) 2014 Massimo Nocentini

Permission is hereby granted, free of charge, to any person obtaining a copy
of this software and associated documentation files (the "Software"), to deal
in the Software without restriction, including without limitation the rights
to use, copy, modify, merge, publish, distribute, sublicense, and/or sell
copies of the Software, and to permit persons to whom the Software is
furnished to do so, subject to the following conditions:

The above copyright notice and this permission notice shall be included in all
copies or substantial portions of the Software.

THE SOFTWARE IS PROVIDED "AS IS", WITHOUT WARRANTY OF ANY KIND, EXPRESS OR
IMPLIED, INCLUDING BUT NOT LIMITED TO THE WARRANTIES OF MERCHANTABILITY,
FITNESS FOR A PARTICULAR PURPOSE AND NONINFRINGEMENT. IN NO EVENT SHALL THE
AUTHORS OR COPYRIGHT HOLDERS BE LIABLE FOR ANY CLAIM, DAMAGES OR OTHER
LIABILITY, WHETHER IN AN ACTION OF CONTRACT, TORT OR OTHERWISE, ARISING FROM,
OUT OF OR IN CONNECTION WITH THE SOFTWARE OR THE USE OR OTHER DEALINGS IN THE
SOFTWARE.
\end{verbatim}

    \subsection{Project hosting}
    All the content of this project has been versioned in a \emph{Git} repository:\\
    \url{https://github.com/massimo-nocentini/quantitative-systems-analysis-exam}.\\

    Actually, the complete folder with Snort log cleaned files and the
    folder containing eps files aren't under version control since too
    large. To make it easier to download them we provide a
    \emph{Makefile} in \emph{pharo-smalltalk} directory under the
    project root, with the following rules:
    \begin{itemize}
    \item \textbf{download-pharo}, will download a complete Pharo environment, about 20 MB;
    \item \textbf{run-pharo}, will run the downloaded Pharo image;
    \item \textbf{download-eps}, will download all eps files necessary
      for compiling this \TeX document, positioning the folder in the
      correct position, about 10 MB;
    \item \textbf{download-snort-logs}, will download the entire snort
      logs tree with the comprehensive set of eps not reported in this
      document, about 400 MB;
    \item \textbf{clean}, will remove all \TeX compilation output file
      and all files generated by running the Smalltalk test suite for
      our implementation, ie all plots and data files;
    \item \textbf{clean-all}, will remove eps and snort logs folders
      recursively, be sure to request this rule.
    \end{itemize}

    \newpage

    All code developed in \emph{Pharo Smalltalk} is freely available
    at
    \url{http://smalltalkhub.com/#!/~MassimoNocentini/QuantitativeSystemsAnalysisExam}
    and can be loaded directly via \emph{Monticello} smalltalk browser
    with the following message:
    \begin{adjustwidth}{-2cm}{}
\begin{verbatim}
MCHttpRepository
	location: 'http://smalltalkhub.com/mc/MassimoNocentini/QuantitativeSystemsAnalysisExam/main'
	user: ''
	password: ''
\end{verbatim}
    \end{adjustwidth}    
    In order to use our implementation the package \emph{CommandShell}
    is required: to load it into the image, open the world menu (click
    on an empty point on the image) then navigate to \emph{Tools} and
    then \emph{Configuration Browser}, after that select the
    \emph{CommandShell} in the list and then click \emph{Install
      Stable Version}.
      
    \begin{thebibliography}{9}

    \bibitem{SNORT}
      SNORT Team,
      \url{https://www.snort.org/}.

    \bibitem{SNORT-manual}
      SNORT Team,
      \url{http://manual.snort.org/}.

    \bibitem{vmstat}
      Free Software Foundation,
      \url{http://www.freebsd.org/cgi/man.cgi?query=vmstat}.

    \bibitem{sed}
      Free Software Foundation,
      \url{http://www.freebsd.org/cgi/man.cgi?query=sed}

    \bibitem{gnuplot} Free Software
      Foundation,\url{http://www.gnuplot.info/}

    \bibitem{bondavalli} Andrea Bondavalli, Analisi quantitativa dei
      sistemi critici, March 2011, Progetto Leonardo, Esculapio
      Editore

    \bibitem{weiher-ducasse} Marcel Weiher and Stephane Ducasse,
      Higher Order Messaging, Dynamic Languages Symposium (DLS) '05,
      October 18, 2005, San Diego, CA, USA


    \end{thebibliography}

\end{document}
